\documentclass[11pt]{amsart}
\usepackage{amsfonts}
\usepackage{amssymb}
\usepackage{graphicx}
\usepackage{url}
\usepackage[margin=1in]{geometry}

\title{Hand in Module 2}
\author{Marius Haakonsen, Ole K Larsen}

\begin{document}

\maketitle

\section{Task 1:}

\textbf{Exercise A:}\\
The both of us have already played Tetris, so we'll skip this part.  \\

\textbf{Exercise B:} \\

Implementing the function void Board::reduce() to remove the lines when completed. \\

Looping over row number i, from top to bottom. \\
\begin{verbatim}

	void Board::reduce() {
	    for(int i = 3; i < 19; i++) {

\end{verbatim}

Defining variables to use while looping over j number of columns. \\
\begin{verbatim}

	        int count = 0;
	        int tilecount = 0;
	        for(int j = 1; j < 11; j++) {
	            if (tiles[j][i] != sf::Color::Black) {
	                count++;

\end{verbatim}

If all tiles in row number 'i' is inequal to the color black, the program loops from that
row and upwards, setting the current row to be equal to the row above, giving the impression that the rows "falls down". \\
\begin{verbatim}

	              if(count == 10) {
	                  tilecount = i;
	                  for(int k = tilecount; k >= 3; k--) {
	                      for(int j = 1; j < 11; j++) {
	                          tiles[j][k] = tiles[j][k-1];
	                      }
	                  }
	                  break;
	              }
\end{verbatim}

As a little coding-exercise from out part, we also added the functionality of pressing spacebar to make the shapes move all the way down until it intersects with other shapes. We also added a score.

\textbf{Exercise C:} \\

To solve this problem we created a header file called "polyminos.h" and included it in the main program. This file will be available for users to create their own polyminos.
The current setup is harcoded to be limited to 6 different polyminos. The main program calls a function with three parameters from the header file. We included some instructions
in the very top of the header file on how to customize the shapes. \\

Here's an \underline{example} of one of the shapes in the .h file.
Explanation of the parameters is in the comments.
\begin{verbatim}

int polyminos(int x, int y, int z) {

	switch (x) {						//x = shape number

	case 0:

		switch (y) {				//y = tile number, totalling at 6
		case 0:
			if (z == 1)				//if z = 1 means its the x coordinate of the tile you want
				return 2;				//in this case the tile (2, 0) will be coloured
			return 0;
		case 1:
			if (z == 1)				//In this case tile (2,1) will be coloured
				return 2;
			return 1;
		case 2:
			if (z == 1)
				return 2;
			return 2;
		case 3:
			if (z == 1)
				return 2;
			return 3;
		case 4:
			if (z == 1)
				return 2;
			return 4;
		case 5:
			if (z == 1)
				return 2;
			return 5;
		}					//All this sums up to a shape that is 1x6 in size, which is a straigth line

		\end{verbatim}

	Instead of moving the entire class to the .h file, which would also be a possible approach to the problem, we wanted to make a solution that was easier to grasp,
	and remove the users ability to screw up the functions in the class etc.

\section{Task 2:}

\textbf{Exercise A:}\\

How was this puzzle created? \\
	Puzzling.stackexchange.com was utilized to get the correct specifications of the puzzle
	and inspiration for the statements to be made by the three people in the encounter. \\

Knight: Always tells the truth.\\
Knave: 	Always tells a lie.\\
Spy: 		Tells either the truth or a lie. \\

The puzzle involves encountering three different people, person A, B and C. \\

They all have their own statements: \\

	A: "C is the knight. hot dog"\\
	B: "A is the knight."\\
	C: "B is the knave." \\

Who is the knight, who is the knave, and who is the spy among the three? \\

To reach the solution of this, the statements are put in a table. \\
\begin{tabular}{ |p{3cm}||p{3cm}|p{3cm}|p{3cm}|  }
 \hline
 \multicolumn{4}{|c|}{Knight, Knave and Spy} \\
 \hline
 Combinations & Is it true? & Would it be said? & Solution  \\ A  B  C   &   A  B  C  &   A  B  C & A  B  C \\
 \hline
	0   1   x  	& 0	0	0   & 1 0 x &  \\
	1   0   x		& 0	1	1  	& 0 0 x &  \\
	0   x   1 	&	1	0	0 	& 0 x 0 &  \\
	1   x   1  	&	0	1	0 	& 0 x 1 &  \\
	x   0   1		& 1	0	1  	& x 1 1 & 1\\
	x   1   0		& 0	0	0  	& x 0 0 &  \\
 \hline
\end{tabular}

The solution is found by asking the two following questions: 'Is it true?' and 'Would it be said?'. \\

Based on the combinations of these columns, and based on whether or not the person actually tells the truth or lies,
we find the correct combination of knight knave and spy. The next to last row is the only possible solution,
as shown by the 'x 1 1' outcome. This means that C is the knight, B is the knace and A is the spy. \\


\textbf{Exercise B:} \\
Reformulated statements, giving the same answer:\\

	A: "C may tell the truth."\\
	B: "C is not a knight."\\
	C: "B may tell a lie." \\


\begin{tabular}{ |p{3cm}||p{3cm}|p{3cm}|p{3cm}|  }
	\hline
	\multicolumn{4}{|c|}{Knight, Knave and Spy} \\
	\hline
	 Combinations & Is it true? & Would it be said? & Solution  \\ A  B  C   &   A  B  C  &   A  B  C & A  B  C \\
	\hline
	  0   1   x  	& 1	1	0   & 0 1 x	&  \\
	  1   0   x		& 1	1	1  	& 1 0 x &  \\
	  0   x   1 	&	1	0	1 	& 0 x 1	&  \\
	  1   x   1  	&	0	1	1 	& 0 x 0	&  \\
	  x   0   1		& 1	0	1  	& x 1 1	& 1\\
	  x   1   0		& 0	1	0  	& x 1 0 &  \\

	\hline
\end{tabular}


\section{Task 3:}

The forums at Stackoverflow.com was used for the solution to this task. \\

The program is written entirely using C++, as we are most familiar with this syntax.
In addition it was limited time from the lecture to when a complete
implementation using GeCode would have had to be finished, this also made it more practical to use C++. \\

\textbf{Exercise A:} \\

in 'main()' the function 'solved()' is called on the grid with an unknown number of
spots set to 0. 'clear()' returns solved as true if the position != 0. \\
'isSafe()' utilizes three functions to make sure the placement of the number i (in the loop) is
not already placed in the relevant row, column and block. A block is 1 of 9 3x3 sized grids within
the 9x9 grid. \\


\begin{verbatim}

	if (!clear(grid, r, c)) { return true; }
	for (int i = 1; i <= 9; i++) {
			if (isSafe(grid, r, c, i)) { grid[r][c] = i;
					if (solve(grid)) { return true; }
					grid[r][c] = 0;
			}


\end{verbatim}


\textbf{Exercise B:} \\

To create a new puzzle, a function that removes an unknown amount of numbers is
cast on the completed grid inside 'main()'. A random number engine is used to create a number
between 0 and 3. The function loops over every row and coloumn, and if the random number
equals 0 or 1, then the position is set to 0. \\
This function may result in an un-solveable game, but this very rarely happens. \\
The if statement can be changed to create easier or harder puzzles, depending on how many
numbers are to be removed from board. \\

\begin{verbatim}

std::uniform_int_distribution<> dist(0,4);
x = dist(generator);
if (x==0 || x==1) { grid[r][c] = 0; }

\end{verbatim}

\section{\textbf{Code Appendix:}}

\textbf{Task 1 main program:}

\begin{verbatim}


#include <SFML/Graphics.hpp>
#include <cmath>
#include <cstdlib>
#include <iostream>
#include <string>
#include "polyminos.h"

using namespace std;

//
// Tetris
//



class Shape
{
public:
	sf::Color tiles[6][6];
	sf::Vector2i pos;

	// times since last downward movement
	float time;

	Shape();

	// reinitialise the shape: move to top and change shape
	void init();

	// move downwards once per second
	void update(float dt);

	// render the shape
	void draw(sf::RenderWindow& w);

	// rotate the shape
	void rotateLeft();
	void rotateRight();
};

void Shape::rotateLeft()
{
	sf::Color tmp[6][6];
	for (int i = 0; i < 6; i++) {
		for (int j = 0; j < 6; j++) {
			tmp[i][j] = tiles[j][5 - i];
		}
	}
	for (int i = 0; i < 6; i++) {
		for (int j = 0; j < 6; j++) {
			tiles[i][j] = tmp[i][j];
		}
	}
}

void Shape::rotateRight()
{
	rotateLeft();
	rotateLeft();
	rotateLeft();
}

Shape::Shape()
{
	init();
}

void Shape::init()
{
	// move to top and reset timer
	pos.y = 0;
	pos.x = 4;
	time = 0.0f;

	// fill with black tiles
	for (int i = 0; i < 6; i++) {
		for (int j = 0; j < 6; j++) {
			tiles[i][j] = sf::Color::Black;
		}
	}

	int shapenr = 0;						// rand() % 7;

	for (int i = shapenr; i < 6; i++) {
		tiles[polyminos(shapenr, i, 1)][polyminos(shapenr, i, 2)] = sf::Color::White;
	}



}

void Shape::update(float dt)
{
	time += dt;
	if (time > 1) {
		time = 0;
		pos.y += 1;
	}
}

void Shape::draw(sf::RenderWindow& w)
{
	sf::CircleShape s;
	s.setRadius(8);
	s.setOrigin(8, 8);
	for (int i = 0; i < 6; i++) {
		for (int j = 0; j < 6; j++) {
			if (tiles[i][j] != sf::Color::Black) {
				s.setFillColor(tiles[i][j]);
				s.setPosition(sf::Vector2f(pos.x * 16 + 16 * i + 100, pos.y * 16 + 16 * j + 100));
				w.draw(s);
			}
		}
	}
}

class Board
{
public:
	sf::Color tiles[12][20];

	Board();

	// add a shape to the board
	void add(Shape& shape);

	// check if a shape intersects with the board
	bool intersect(Shape& shape);

	// remove full lines - should be implemented by you
	void reduce();

	// render board
	void draw(sf::RenderWindow& w);
};

void Board::reduce()
{

}

bool Board::intersect(Shape& shape)
{
	bool intersect = false;
	for (int i = 0; i < 6; i++) {
		for (int j = 0; j < 6; j++) {
			if (tiles[i + shape.pos.x][j + shape.pos.y] != sf::Color::Black &&
				shape.tiles[i][j] != sf::Color::Black)
				intersect = true;
		}
	}
	return intersect;
}

void Board::draw(sf::RenderWindow& w)
{
	sf::CircleShape s;
	s.setRadius(8);
	s.setOrigin(8, 8);
	for (int i = 0; i < 12; i++) {
		for (int j = 0; j < 20; j++) {
			s.setFillColor(tiles[i][j]);
			s.setPosition(sf::Vector2f(16 * i + 100, 16 * j + 100));
			w.draw(s);
		}
	}
}

void Board::add(Shape& shape)
{
	for (int i = 0; i < 6; i++) {
		for (int j = 0; j < 6; j++) {

			if (shape.tiles[i][j] != sf::Color::Black) {
				tiles[i + shape.pos.x][j + shape.pos.y] = shape.tiles[i][j];

			}

		}

	}


}





Board::Board()
{
	// fill with black
	for (int i = 0; i < 20; i++) {
		for (int j = 0; j < 12; j++) {
			tiles[j][i] = sf::Color::Black;
		}
	}

	// boundary
	for (int i = 0; i < 12; i++) {
		tiles[i][19] = sf::Color::Red;
	}
	for (int i = 0; i < 19; i++) {
		tiles[0][i] = sf::Color::Red;
		tiles[11][i] = sf::Color::Red;
	}
}


int main()
{
	sf::RenderWindow window(sf::VideoMode(512, 512), "Tetris");
	sf::Font font; font.loadFromFile("joystix monospace.ttf");
	sf::Text score; score.setString("Score: ");
	score.setCharacterSize(30);
	score.setFont(font);
	score.setPosition(300, 200);
	sf::Clock clock;
	clock.restart();
	int score1 = 0;
	sf::Text score2;
	score2.setFont(font);
	score2.setCharacterSize(50);
	score2.setPosition(320, 250);

	Shape shape;

	Board board;

	// run the program as long as the window is open
	while (window.isOpen())
	{
		// check all the window's events that were triggered since the last iteration of the loop
		sf::Event event;

		while (window.pollEvent(event))
		{
			// "close requested" event: we close the window
			if (event.type == sf::Event::Closed)
				window.close();


			if (event.type == sf::Event::KeyPressed) {
				if (event.key.code == sf::Keyboard::Left) {
					shape.pos.x -= 1;
					if (board.intersect(shape)) {
						shape.pos.x += 1;
						cout << "intersect left" << endl;
					}
				}
				if (event.key.code == sf::Keyboard::Right) {
					shape.pos.x += 1;
					if (board.intersect(shape)) {
						shape.pos.x -= 1;
						cout << "intersect right" << endl;
					}
				}
				if (event.key.code == sf::Keyboard::Down) {
					shape.pos.y += 1;
					if (board.intersect(shape)) {
						shape.pos.y -= 1;
						cout << "intersect down" << endl;
					}
				}
				if (event.key.code == sf::Keyboard::Up) {
					shape.rotateLeft();
					if (board.intersect(shape)) {
						shape.rotateRight();
						cout << "intersect rotate" << endl;
					}
				}
				if (event.key.code == sf::Keyboard::Space) {
					while (!board.intersect(shape)) {
						shape.pos.y += 1;
						cout << "fast down" << endl;
					}
				}
			}
		}

		float dt = clock.restart().asSeconds();

		shape.update(dt);

		int k = 20 - 1;
		for (int i = 20; i > 0; i--) {
			int count = 0;
			for (int j = 1; j < 12; j++) {
				if (board.tiles[j][i] != sf::Color::Black && board.tiles[j][i] != sf::Color::Red)
					count++;
				board.tiles[k][j] = board.tiles[i][j];
			}
			if (count == 10) {
				for (int d = 1; d < 11; d++)
					board.tiles[d][i] = sf::Color::Black;
				for (int f = i; f > 0; f--)
					for (int w = 1; w < 11; w++)
						board.tiles[w][f] = board.tiles[w][f-1];
				score1 += 10;
			}

		}
		score2.setString(to_string(score1));


		if (board.intersect(shape)) {
			shape.pos.y -= 1;
			board.add(shape);
			board.reduce();
			shape.init();
			if (board.intersect(shape)) {
				cout << "GAME OVER" << endl;
			}
		}

		window.clear(sf::Color::Black);

		board.draw(window);
		shape.draw(window);
		window.draw(score);
		window.draw(score2);
		window.display();
	}

	return 0;
}





\end{verbatim}



\end{document}
