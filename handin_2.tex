\documentclass[11pt]{amsart}
\usepackage{amsfonts}
\usepackage{amssymb}
\usepackage{graphicx}
\usepackage{url}
\usepackage[margin=1in]{geometry}

\title{Hand in Module 2}
\author{Marius Haakonsen, Ole K Larsen}

\begin{document}

\maketitle

\section{Task 1:}

Exercise A  \\
The both of us have already played Tetris, so we'll skip this part.  \\

Exercise B \\

Implementing the function void Board::reduce() to remove the lines when completed. \\

Looping over row number i, from top to bottom. \\
\begin{verbatim}

	void Board::reduce() {
	    for(int i = 3; i < 19; i++) {

\end{verbatim}

Defining variables to use while looping over j number of columns. \\
\begin{verbatim}

	        int count = 0;
	        int tilecount = 0;
	        for(int j = 1; j < 11; j++) {
	            if (tiles[j][i] != sf::Color::Black) {
	                count++;

\end{verbatim}

If all tiles in the row is set to black, looping upward begins, moving each tile
in each row one down. \\
\begin{verbatim}

	                if(count == 10) {
	                    tilecount = i;
	                    for(int k = tilecount; k >= 3; k--) {
	                        for(int j = 1; j < 11; j++) {
	                            tiles[j][k] = tiles[j][k-1];
	                        }
	                    }
	                    break;
	                }

\end{verbatim}


\section{Task 2:}

How was this puzzle created?
	Puzzling.stackexchange.com was utilized to get the correct specifications of the puzzle,
	and inspiration for the statements to be made by the three people in the encounter. \\

Knight: Always tells the truth.
Knave: 	Always tells a lie.
Spy: 		Tells either the truth or a lie. \\

The puzzle involves encountering three different people, person A, B and C. \\

They all have their own statements:

	A: "I am the only knight."
	B: "Me and A are knights."
	C: "B is a knave." \\

Who is the knight, who is the knave, and who is the spy among the three? \\


Reformulated statements:

	A: "B and C are not knights"
	B: "C is the only one not telling the truth."
	C: "B is not a knight." \\


\begin{center}
\begin{tabular}{||Comb. True? Said? Event||}
\hline
Col1 & Col2 & Col2 & Col3 \\ [0.5ex]
\hline\hline
1 & 6 & 87837 & 787 \\
\hline
2 & 7 & 78 & 5415 \\
\hline
3 & 545 & 778 & 7507 \\
\hline
4 & 545 & 18744 & 7560 \\
\hline
5 & 88 & 788 & 6344 \\ [1ex]
\hline
\end{tabular}
\end{center}

\section{Code Appendix:}

\begin{verbatim}



\end{verbatim}

\end{document}
