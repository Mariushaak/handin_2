\documentclass[11pt]{amsart}
\usepackage{amsfonts}
\usepackage{amssymb}
\usepackage{graphicx}
\usepackage{url}
\usepackage[margin=1in]{geometry}

\title{Hand in Module 2}
\author{Marius Haakonsen, Ole K Larsen}


\begin{document}

\maketitle

\section{Task 1:}

Exercise A:  \\
The both of us have already played Tetris, so we'll skip this part.  \\




\section{Task 2:}

How was this puzzle created?
	Puzzling.stackexchange.com was utilized to get the correct specifications of the puzzle,
	and inspiration for the statements to be made by the three people in the encounter. \\

Knight: Always tells the truth.
Knave: 	Always tells a lie.
Spy: 	Tells either the truth or a lie. \\

The puzzle involves encountering three different people, person A, B and C. \\

They all have their own statements:
	A: "C is a knave."
	B: "I am a knight."
	C: "B is not a knight." \\

Who is the knight, who is the knave, and who is the spy among the three? \\


\end{document}
